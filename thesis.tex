% An Applied Mathematics/Biology Thesis
% By Jonathan Niles

\documentclass[phd,tocprelim]{cornell}

% packages
\usepackage{natbib}
\usepackage{geometry}
\usepackage{longtable}
\usepackage{ifpdf}
\usepackage{amsmath}
\usepackage{appendix}
\usepackage[hidelinks]{hyperref}
\usepackage[toc]{glossaries}

% Set tolerance for box model
\tolerance=9999

% New Commands, refreshers
%\renewcommand{\caption}[1]{\singlespacing\hangcaption{#1}\normalspacing}
\renewcommand{\topfraction}{0.85}
\renewcommand{\textfraction}{0.1}
\renewcommand{\floatpagefraction}{0.75}
\newenvironment{definition}[1][Definition]{\begin{trivlist}
  \item[\hskip \labelsep {\bfseries #1}]}{\end{trivlist}}

\title{Machinery for Chromosomal Fragility and Translocation Frequency}
\author{Jonathan Niles}
\conferraldate{June}{2014}
\degreefield{Bachelor of Science}
\copyrightholder{Jonathan Niles}
\copyrightyear{2014}


% Acronym Definitions
\newacronym{dna}{DNA}{Deoxyribonucleic Acid}
\newacronym{rna}{RNA}{Ribonucleic Acid}
\newacronym{geo}{GEO}{Gene Expression Omnibus}
\newacronym{ncbi}{NCBI}{National Center for Biotechnology Information}
\newacronym{sra}{SRA}{Sequence Read Archive}
\newacronym{tad}{TAD}{Topologically Associating Domain}
\newacronym{lad}{LAD}{Lamina Associating Domain}

% Glossary definitions
\newglossaryentry{karyotype}{%
  name={karyotype},
  description={A photomicrograph of chromosomes arranged according to a standard classification.}%
}

\newglossaryentry{polymer}{%
  name={polymer},
  description={A substance that has a molecular structure consisting chiefly or entirely of a large number of similar units bonded together, e.g., many synthetic organic materials used as plastics and resins.}
}

\newglossaryentry{restriction enzyme}{%
  name={restriction enzyme},
  description={An enzyme that restricts, or performs double-strand cut at a specific DNA sequence motif.}
}

\newglossaryentry{ligation}{%
  name={ligation}
  description={The joining of two DNA strands or other molecules by a phosphate ester linkage.}
}

% Make a glossary using the glossaries package
\makeglossaries%

\begin{document}

\maketitle
\makecopyright%

\begin{abstract}
Chromosomal rearrangements and sequence translocations are
canonical hallmarks of cancerous cells.  The canonical models
for these chromosomal breakage events involve spatially
independent molecular mechanisms of double-strand break
repair and non-homologous end joining.  The recent development of high
throughput techniques to interrogate genome-wide chromatin
conformation allows for the spatial dynamics of chromosomal
breakage to be examined and correlated with translocations
frequencies and other genomic rearrangements.  I present
evidence for translocaction frequency based on distances between
translocating sites in 3D space.  I also propose experiments
to further elucidate the interplay between distance and
genomic rearrangements in cells.
\end{abstract}

%\begin{dedication}
%This document is dedicated to Miriam Paul Fountain.
%\end{dedication}

\begin{acknowledgements}
I would like to thank my faculty sponsors for their patience,
dedication, and sympathy: Dr. Tyrone Ryba, Dr. Patrick McDonald,
and Dr. David Gillman.
\end{acknowledgements}

\contentspage%
\tablelistpage%
\figurelistpage%

\normalspacing\setcounter{page}{1} \pagenumbering{arabic}
\pagestyle{cornell} \addtolength{\parskip}{0.5\baselineskip}

\chapter{Introduction}

The human genome is a three billion nucleotide string that encodes the
the vast instruction set for the cells growth, function, and death.  The cell
programme, however, is dictated not only by the instruction set, but also
how the instructions are read, compiled, and exectuted, and is influenced
by a myriad of factors in addition to the primary sequence.  Increasingly,
modern cancer research is indicating this epigenetic landscape plays strategic
role in determining cell fate.  Understanding the landmark epigenetic features,
the conditions which bring about their development, and the impact they have on
the cell is of paramount importance and difficultly.

Perhaps the most common and least understood effects of genomic instability are
translocations between disperate genomic regions.  Recently, several proposed
molecular mechanisms have gained traction in explaining how translocations arise
but why they arise is a question still without an answer.

Here, I investigate genome-wide interaction data to and show that proximity in
genomic space strongly correlates with translocation probability.






\chapter{The Nuclear Architecture}

\section{Building from the Basics}

It is difficult to overstate the importance of the nuclear topology in
every aspect of a eukaryotic cell's life.  The morass of nucleic acids and
proteins packed tightly inside the nuclear envelope contains the entireity of
the information necessary to completely duplicate the cell.

The sheer complexity of the nucleus requires us to investigate the biological
phenomena at multiple layers of granularity.  Ironically, the finest resolution
of the nucleus is also the most well studied: the primary sequence of nucleic
acid bases which compose the DNA molecule itself.  One level of abstraction
above the primary sequence considers the interactions of the primary sequence
with architectural proteins such as histones, and the secondary structures that
result from these couplings.  Even higher up the abstraction tree lies
\textit{epigenetics} (literally `above genetics'\cite{dictepi2014}), a loose
classification for secondary modifications that can be heritible or characteristic
of a cell state.  Finally, chromosomes and chromosomal territories compose the
highest and most recently characterized levels of nuclear architecture.  The
construction of this stratum provides researchers with a conceptual tool with
which to model and consider the topology of the nuclear contents.

\subsection{Foundations: The  Primary Sequence}

The fundemental informational unit of the nucleus is the nucleotide, often
called a base or base pair when joined in a molecule or strand.  Eukaryotic
cells have a principle four character alphabet of nucleotides, distinguished
by their \textit{nucleobases}: adenine, cytosine, guanine, and thymine.  These
nucleotides are affixed to a phosphate backbone into a long, coiling polymer
chain.  The famous publication of the structure of DNA in 1953 by researchers
Watson and Crick determined that these polymers exist in the nucleus as double
helices\cite{watson1953}, with hydrogen bonds holding the opposing strands
together.  Importantly, adenine and thymine form a two hydrogen bond
complimentary pair, while guanine and cytosine form three bonds.  In this way,
one particular side of the helix can determine the other side (in most circumstances)
and deviations from this pattern can lead to mutations of various types\cite{cox2012}.

One cannot ignore the consituents of the primary sequence, particular when
considering macroscopic properties of the genome.  Indeed, the order and
nucleotide content can change the local flexibility and both directly (by
forming binding sites) and indirectly (by supporting the binding of histones)
modules the binding of important architectural proteins\cite{travers2004}.
Although a DNA molecule is a helical structure, it is an intrinsically
flexible molecule.  A naive calculation reveals that the human genome is
roughly two meters in length%
\footnote{%
  A diploid human genome consists of two copies of the \~{}$5.2MB$ DNA molecule,
  based off estimates from the NCBI human genome reference build 37.
  Each base pair is assumed to be roughly $0.34\times10^{-9}m$.
  $2(0.34 \times 10^{-9})(3.2 \times 10^9) = 2.176$
}
, yet the entire genome fits into a nuclear volume of \~{}$374^3\mu{}m$
\cite{marks2011}.  Furthermore, loops as small as 100 bases have been
observed in enhancer and repressor activation pathways\cite{wong2008}, and
can form in the absense of proteins\cite{vafabakhsh2012}.  Altogether,DNA's
inherent flexibility provides a foundation for the formation of many complex
higher order chromatin structures.

\subsection{Epigenetics And Chromatin Modeling}

Where the primary sequence furnishes favorable properties for the formation of
looping and bound structures structures, the epigenetic landscape determines
the formation, maintenance, and conformation of higher order structures.  The
most ubiquitous class of proteins are the histone architecural proteins,
responsible for non-specifically binding DNA at regular intervals.  Histones
and their composite structure, the nucleosome, are the principle, but not singular,
regulators of chromatin accessibility, shape, and dynamics at the epigenetic
level.

The




\chapter{Chromosomal Conformational Capture and Derivatives}

The field of molecular biology is advancing rapidly, propelled forth by a
surge of automated techniques.  So-called `high-throughput' techniques enable
biologists and bioinformaticians to generate and process massive datasets with
both ease, reliability, and efficiency.  The first reference build of the
human genome is a testament to their success\cite{hgsc2004}.  However,
until recently, investigating chromosomal architecture remained an
exhaustively manual process, relying on flourescent microscopy experiments and
visual inspection by researchers.  It wasn't until 2003, when Job Dekker and
collegues described a technique known as Chromosomal Conformational Capture
(3C) that the study of nuclear architecture gained a high throughput methodology
\cite{dekker2002}. Since then, a family of conformational capture techniques
(most named a variation on `3C') have been developed to interrogate
different types of chromatin interactions and achieve higher throughput.

The original chromosome conformation capture technique developed by Dekker and
collegues provides an average measurement of the juxtaposition frequence between
two specific genomic loci\cite{frase2014}.  The frequency of this measurement over a population
is thought to be indicative of the proximity of the two loci in genomic space.
The first derivative technique on 3C, aptly named Chromosome Conformation Capture-
on-Chip (4C), extended the number contacts assessed from two loci to all
contacts made with a particular genomic loci by assessing contacts on a
whole genome microaray\cite{simonis2006}.  Most recently, the Dekker lab has
developed a technique using biotinylated probes to assess contacts between any
given genomic region\cite{berkum2010}.  The method, known as Hi-C, boasts
the first truly global assessment of all genomic contacts at a given time.

The conceptual idea behind chromosomal conformation capture is remarkably
simple --- to assess nearby regions, the experimenter attempts to fuse
together molecules that are physically close, then determine the interaction
partners of each segment of DNA captured in this fashion.  The 3C proceedure
involves five steps.   Initially,  apopulation of (mammalian) cells is cultured in an appropriate growth medium
until a population size of $\~{}2.5 \times 10^7$ cells is achieved\cite{berkum2010}.
The entire population is treated with formaldehyde, a small chemical commonly
used to fix both small samples for microscopy experiments and large specimens
for organismal analysis.  Importantly, formadehyde is a mutagen known to form
DNA-protein crosslinks\cite{merk1998}.  The global affect of the formaldehyde
treatment is fusing together DNA and protein structures that are nearby in
the genomic space.  Crosslinking the entire cell ensures that the nuclear contents
are fixed before any further experimental intervention, exactly capturing
the growing cell's nuclear structure.  In the second step, the cells are
homogenized and the chromatin is digested by a restriction enzyme (typically
HindIII or Nco1)\cite{berkmun2010}.  Digestion creates two populations of
DNA fragments; though bound to proteins and other DNA sequences, and an unbound
population.  The unbound population is discarded through filtering.  In the
third step, the bound sequences are ligated (joined) in highly diluted
concentrations, to bias towards ligation reactions between DNA strands bound
to the same molecular complex.  The ligation process combines DNA sequences
from different loci into a single sequence.  The fourth step is to reverse
the crosslinks and release the recombined sequences from their complexes.  The
final step comprises quantifying the restriction fragments by PCR using fragments
specific for the population being studied\cite{simonis2007}.

%
% FIXME
%   The 4C protocol is not very well understood.  Please round out this
% knowledge and fill in the section appropriately.
%

In the derivative methods 4C, 5C, and ChIP-loop, the basic protocol remains
unchanged; however, the capture and analysis of fragments varies depending on
the application.  Chromosome Conformation Capture-on-Chip (4C) leverages a
DNA microarray to systematically screen the entire genome in an unbiased
manner for DNA loci that contact each other\cite{simonis2006}.  The microarray
is tiled with probes located $< 100$ base pairs away from a restriction site.
Rather than performing the PCR step from the canonical 3C assay, the restriction
fragments are shortened with a second restriction enzyme, circularized, and
amplified by inverse PCR\@.  The amplified probes are detected on the designed
microarray\cite{simonis2006}.  Chromosome Conformation Capture Carbon-Copy (5C)





ANALYSIS

Chromosome Conformation Capture is a quantitative assay and meaningful
results are gleaned by comparison of relative interation frequencies
between genomic regions. %talk about controls, controls controls




















\chapter{Fragility in the Code}

The first investigations into nuclear architecture were conducted by
examining cell karyotypes.  Despite the development of the karyotype as a
tool for examining nuclear material in the early 20th
century\cite{levitsky1924}, it wasn't until 1954 the number of chromosomes in
a human cell were definitely described\cite{tjio1956}.  Early investigations
into nuclear architecture, even at the course level of a karytype, were
hampered by technical limitations and chromosomal phenomena such as
non-dysjunction and breakage.  It is not surprising that soon after the
initial description the human chromosomal number, Debakan and collegues
characterized common sites were chromosomes would undergo breakage or
translocations.  They termed these regions
\textit{chromosomal fragile sites}\cite{leyden2008}.

Many particularities render studying fragile sites difficult.  The
first difficulty is semantic; chromosomal fragile sites are not precisely
defined in the literature.  When a study is performed that encompases fragile
sites, typically one of three definitions is used: regions that are particularly
sensative to forming gaps or breaks on metaphase chromosomes\cite{glover2005},
sites where chromatin fails to compace under mitosis\cite{leyden2008}, and
nonrandomly distributed loci that exhibit an increased frequency of breakage
under replicational stress\cite{franchitto2013}.  For the purposes of this
discussion, a \textit{fragile site} is a region on the chromosome prone to
forming complex rearrangements, particulary double-strand breaks, repeat
extensions, and translocations, when subjected to replicational stress.  These
rearrangements play a pivitol role in many severly delibitating genetic
diseases.

Fragile sites come in two flavors: common fragile sites (CFS) and rare fragile
sites (RFS).  Fragile sites are classified into a group based on their
prevalence in the population, and the conditions under which their fragility
is induced\cite{leyden2008}.  Common fragile sites are thought to be common
most chromsomes and to all humans, while rare fragile sites may be expressed
in small fraction (less than 5\%) of the population\cite{wells2006}.

These
rearrangements play pivitol roles in severely debilitating genetic diseases
such as Fragile X Syndrome.  All males and an estimated that 60\% of females
with repeat anomalies near the FRM1 gene on the X chromosome suffer
from severe mental handicap due to these alterations\cite{sutherland1995}.
Additionally, fragiles sites are often found rearranged in human
cancers\cite{glover2005}.  Despite these findings, reason fragile site are
fragile is still an unanswered question.

% CFS's may be epigenetically defined. (Molecular profiling of CFS in human fibroblasts
% CFS are conserved based on cell types


\chapter{Methods}

\section{Data Collection}

The Hi-C datasets used in this thesis were obtained from the Gene
Expression Omnibus (GEO)\cite{edgar2002} as Sequence Read Archive (SRA) files.%
\footnote{Accession Number \href{http://www.ncbi.nlm.nih.gov/geo/query/acc.cgi?acc=GSE43070}{GSE43070}.}
IMR90, the human lung fibroblast cell line, was selected for analysis due to
an abundance of high resolution, publically available interaction data.
Additionally, certain common fragile sites serve as frequent breakpoints in
lung cancers\cite{tunca2002}\cite{dhillon2003}.  A small script was written to
recover six experimental replicates from GEO onto a local research server for
analysis.  Additionally, analyzed tracks for DNA methylation at base pair
resolution were obtained from GEO\cite{lister2009}.  The liftOver tool from
the University of California Santa Cruise was used to update the methylation
data from human genome build hg18 to hg19 for meaningful comparison\cite{hinrichs2006}.
For detailed data collection notes, see Appendix I.

The detailed description of the treatment protocol used in creating the Hi-C
library can be found in the original paper\cite{ren2013}.  Briefly, the IMR90
cells were cultured in growth media.

The cell lines were prepared with no treatment according to the protocol
described by Dekker and colleagues\cite{dekker2013}.  In brief, the cells
are washed with a formaldehyde fixation solution to crosslinks DNA and
protein molecules near each other in genomic space.  The cells are
lysed and the nuclear contents are treated with a non-specific restriction
enzyme, \textit{HindIII}.  After restriction, the remaining nuclear material
is washed so that the remaining contents are short DNA sequences or cross
-linked DNA-protein complexes.  The DNA reads are labeled with a biotin
marker and ligated at low concentrations that favor within ligation between
reads bound to the same complex.  The reads are captured and sequenced,
building what is known as a Hi-C library.

%% Regulatory Elements from Ensembl

%% Translocations from COSMIC

%% Genes from Biomart
%% Hg19 from UCSC


\section{Mapping and Alignment}

A Hi-C library is formed largely of \textit{chimeric} DNA sequences,
sequences composed of material from more than one source.  The ligation process
creates novel DNA sequences with a portion of sequence from one region of
the genome and the remaining portion from a different genomic region.  In
order to discover to which two regions each probe corresponds, the probes
must be realigned to the human genome.  Previous methods (\textit{citation needed})
have truncated reads to a fixed cutoff size before realignment with a short
read alignment algorithm.  In a recent paper, Imakaev and collegues
describe an algorithm to repeated algin reads with different truncation sizes
and analyzing reads with the maximum alignment score\cite{imakaev2012}.  We followed
this proceedure to realign the IMR90 Hi-C libraries for six replicates to
the human genome build 19 (CRCh38/hg19), with an average of XX million reads
mapped per replicate.

\section{Normalization of Contact Maps}

Understanding and accounting the biases inherent in a dataset is vital to
producing accurate, biologically relevant analysis.  To this end, several methods
have been proposed identify biases and normalize contact maps between replicates
and experiments\cite{yaffe2011}\cite{hu2012}\cite{yang2014}.  The earliest methods on low
resolution samples did not perform any normalization at all\cite{aiden2009};
however, as higher resolution datasets have come available, better
normalization methods are being explored to identify reproducible,
fine-grained chromatin structures.

Normalizing data from Hi-C experiments is quite challenging.  As previously
discussed, the Hi-C datasets are heterogenous in nature, encorporating interaction encorporating interaction encorporating interaction encorporating
interactions from cells in every position of cell cycle.  As evidence of phenomena,
the interaction matrices produced by the Hi-C experiments have nonzero contact
probabilities between nearly every genomic coordinate\cite{dekker2013}.  This
finding is further reinforced by the observation that single cell Hi-C analysis
shows high variability in chromatin architecture\cite{nagano2013}.  Despite this
variability, existing normalization techniques for Hi-C datasets typically
assume the majority of the cells in the sampled ensemble are in a static
chromatin conformation reflective of \textit{in vivo} cell populations.

One approach to normalization described by Yaffe and colleagues\cite{yaffe2011}
attempts to describe the total interaction frequence between loci as the product
of a true interaction frequency and a set of experimental biases.  These biases
are present in all chromosomal capture techniques.  They include biases due to
PCR in efficiency, read depth per region, sequencing bias, chromatin compaction,
and random background collisions during
ligation\cite{benner2014} \cite{dekker2006}.



%% Data Validation

\section{Analysis}

%% Iterative Correction and Eigenvalue Expansion

%% Visualization Tools
%% Replication Timing Analysis..?




\chapter{Results \& Discussion}


%
% APPENDIX
%

\appendix
\appendixpage%
\addappheadtotoc%
\chapter{Data Collection}

Methylation data was downloaded from the Salk Institute for Biology Studies in
two files from two biological replicates.  Each file contained $~{}600$ million
reads from the

\begin{center}
  \begin{tabular}{lcr}
    \hline
    Replicate & Hg18 Reads & Hg19 Reads & Unlifted \\
    \hline
    1 & 563,354,527 & 563,071,323 & 566,408 \\
    2 & 620,520,572 & 620,227,842 & 585,460 \\
    \hline
  \end{tabular}
\end{center}


\chapter{Iterative Alignment of Probes}

Probes were aligned to the human genome build hg19 using proceedures outlined by
Imakaev and collegues\cite{imakaev2012}.  The chimeric nature of the reads
requires that probes be aligned iteratively, starting from a small, truncated
region from the beginning of the read, mapping this truncated area, increasing
the truncation size and recursing a fixed number of steps or until the alignment
scores become sufficiently poor.  Due to the large number of reads requiring
alignment, we opted to use a fixed truncation length (based on sequence length)
and four steps in the iterative alignment protocol.  The calculation
for the truncation and step size can be found in the the iterativeMapping.py
script provided in the Appendix: Code.  Most reads were 100 base pairs, resulting
in an initial truncation length of 28 base pairs, and step size of 18 base pairs.

Using the mapping functionality from the hiclib python package\cite{imakaev2012},
sequences from the six experimental replicates were realigned to the genome.  The
alignment employed the fast Bowtie2 alignment algorithm\cite{langmead2012}.  Once
aligned, the probes were stored as an interaction matrix in the high performance
HDF5\cite{hdf5} data format, a total of 25Gb for all replicates.

Statistics for iterative alignment are given below:

\begin{center}
  \begin{table}
    \begin{tabular}{l l}
    Total Reads & 2,124,453,478 \\
    Total DS Reads & 1,422,870,270 \\
    Valid Pairs & 713,897,554 \\
    Filtered Reads & 457,298,174 \\
    Percent \textit{trans} Reads & 49.42\% \\
    \end{tabular}
  \end{table}
\end{center}


\chapter{Validation of Data Correctness}

In order to make meaningful comparisons between datasets (replicates,
in this case), we must show that some degree of relationship exists between
the datasets and comparisons or combinations of the data from disperate sets
are valid to a degree of uncertainty.  It is also essential to understand if the
experimental replicates indeed managed to replicate the conditions of the primary
experiment, or if experimental errors prevent the comparison between replicates.

Spearman's Rank Correlation Coefficient (denoted by the Greek letter \textit{rho}
$\rho$) is a nonparametric measurei of association between two variables.
Spearman's coefficient assumes some monotonic relationship between variables,
rather than a linear relationship (as in Pearson's), making it appropriate
to compare the IM90 interaction datasets.  The formula for Spearman's $\rho$ is
given as follows:

$
\rho = 1 - \frac{\sum_{i=1}{n}(d_i^2)}{n(n^2 - 1)}
$

where $\rho$ is the correlation coefficient taking values between $-1$ and $+1$,
$d_i = x_i - y_i$ where $x_i, y_i$ are ranks derived from the raw scores $X$ and
$Y$ respectively.

The first replicate IMR90 interation dataset was labeled the primary dataset
and the remaining five were compared using Spearman's Rank Correlation.  The
results are given in Table X.

%% Example data for now.
\begin{table}
  \begin{tabular}{|c|*{6}{c|}}
    \textbf{R1} & 0.83 & 0.77 & 0.77 & 0.75 & 0.71 \\ \hline
    0.83 & \textbf{R2} & 0.82 & 0.83 & 0.79 & 0.75 \\ \hline
    0.77 & 0.82 & \textbf{R3} & 0.77 & 0.74 & 0.73 \\ \hline
    0.77 & 0.83 & 0.77 & \textbf{R4} & 0.75 & 0.71 \\ \hline
    0.75 & 0.79 & 0.74 & 0.74 & \textbf{R5} & 0.69 \\ \hline
    0.71 & 0.75 & 0.73 & 0.71 & 0.69 & \textbf{R6} \\
  \end{tabular}
  \caption{Spearman's $\rho$ across all datasets.}
  \label{tab:correlations}
\end{table}

\chapter{Common Fragile Sites}
This is from\cite{fragsites2001}.

\begin{center}
  \begin{longtable}{|l|l|l|l|}
    \caption[Human Fragile Sites]{Human Fragile Sites
    } \label{grid_hfs} \\ \hline

    \endfirsthead
    {{\bfseries \tablename\ \thetable{} -- continued from previous page}} \\ \hline
    \endhead

    \hline \multicolumn{3}{|r|}{{Continued on next page}} \\ \hline
    \endfoot

    \hline \hline
    \endlastfoot

     Gene Symbol & Chromosome Band Location & Type & Group \\ \hline
     FRA1A  & 1p36     & Common, aphidicolin   & 4 \\
     FRA1B  & 1p32     & Common, aphidicolin   & 4 \\
     FRA1C  & 2p31.2   & Common, aphidicolin   & 4 \\
     FRA1L  & 1p31     & Common, aphidicolin   & 4 \\
     FRA1D  & 1p22     & Common, aphidicolin   & 4 \\
     FRA1M  & p21.3    & Rare, folic acid    & 1 \\
     FRA1E  & 1p21.2   & Common, aphidicolin   & 4 \\
     FRA1J  & 1q12     & Common, 5-azacytidine & 5 \\
     FRA1F  & 1q21     & Common, aphidicolin   & 4 \\
     FRA1G  & 1q25.1   & Common, aphidicolin   & 4 \\
     FRA1K  & 1q31     & Common, aphidicolin   & 4 \\
     FRA1H  & 1q42     & Common, 5-azacytidine & 5 \\
     FRA1I  & 1q44     & Common, aphidicolin   & 4 \\
     FRA2C  & 2p24.2   & Common, aphidicolin   & 4 \\
     FRA2D  & 2p16.2   & Common, aphidicolin   & 4 \\
     FRA2E  & 2p13     & Common, aphidicolin   & 4 \\
     FRA2L  & 2p11.2   & Rare, folic acid    & 1 \\
     FRA2A  & 2q11.2   & Rare, folic acid    & 1 \\
     FRA2B  & 2q13     & Rare, folic acid    & 1 \\
     FRA2F  & 2q21.3   & Common, aphidicolin   & 4 \\
     FRA2K  & 2q22.3   & Rare, folic acid    & 1 \\
     FRA2G  & 2q31     & Common, aphidicolin   & 4 \\
     FRA2H  & 2q32.1   & Common, aphidicolin   & 4 \\
     FRA2I  & 2q33     & Common, aphidicolin   & 4 \\
     FRA2J  & 2q37.3   & Common, aphidicolin   & 4 \\
     FRA3A  & 3p24.2   & Common, aphidicolin   & 4 \\
     FRA3B  & 3p14.2   & Common, aphidicolin   & 4 \\
     FRA3D  & 3q25     & Common, aphidicolin   & 4 \\
     FRA3C  & 3q27     & Common, aphidicolin   & 4 \\
     FRA4A  & 4p16.1   & Common, aphidicolin   & 4 \\
     FRA4D  & 4p15     & Common, aphidicolin   & 4 \\
     FRA4B  & 4q12     & Common, BrdU          & 6 \\
     FRA4C  & 4q31.1   & Common, aphidicolin   & 4 \\
     FRA5E  & 5p14     & Common, aphidicolin   & 4 \\
     FRA5A  & 5p13     & Common, BrdU          & 6 \\
     FRA5B  & 5q15     & Common, BrdU          & 6 \\
     FRA5D  & 5q15     & Common, aphidicolin   & 4 \\
     FRA5F  & 5q21     & Common, aphidicolin   & 4 \\
     FRA5C  & 5q31.1   & Common, aphidicolin   & 4 \\
     FRA5G  & 5q35     & Rare, folic acid    & 1 \\
     FRA6B  & 6p25.1   & Common, aphidicolin   & 4 \\
     FRA6A  & 6p23     & Rare, folic acid    & 1 \\
     FRA6C  & 6p22.2   & Common, aphidicolin   & 4 \\
     FRA6D  & 6q13     & Common, BrdU          & 6 \\
     FRA6G  & 6q15     & Common, aphidicolin   & 4 \\
     FRA6F  & 6q21     & Common, aphidicolin   & 4 \\
     FRA6E  & 6q26     & Common, aphidicolin   & 4 \\
     FRA7B  & 7p22     & Common, aphidicolin   & 4 \\
     FRA7C  & 7p14.2   & Common, aphidicolin   & 4 \\
     FRA7D  & 7p13     & Common, aphidicolin   & 4 \\
     FRA7A  & 7p11.2   & Rare, folic acid    & 1 \\
     FRA7J  & 7q11     & Common, aphidicolin   & 4 \\
     FRA7E  & 7q21.2   & Common, aphidicolin   & 4 \\
     FRA7F  & 7q22     & Common, aphidicolin   & 4 \\
     FRA7G  & 7q31.2   & Common, aphidicolin   & 4 \\
     FRA7H  & 7q32.3   & Common, aphidicolin   & 4 \\
     FRA7I  & 7q36     & Common, aphidicolin   & 4 \\
     FRA8B  & 8q22.1   & Common, aphidicolin   & 4 \\
     FRA8A  & 8q22.3   & Rare, folic acid    & 1 \\
     FRA8C  & 8q24.1   & Common, aphidicolin   & 4 \\
     FRA8E  & 8q24.1   & Rare, distamycin A  & 2b \\
     FRA8D  & 8q24.3   & Common, aphidicolin   & 4 \\
     FRA9A  & 9p21     & Rare, folic acid    & 1 \\
     FRA9C  & 9p21     & Common, BrdU          & 6 \\
     FRA9F  & 9q12     & Common, 5-azacytidine & 5 \\
     FRA9D  & 9q22.1   & Common, aphidicolin   & 4 \\
     FRA9B  & 9q32     & Rare, folic acid    & 1 \\
     FRA9E  & 9q32     & Common, aphidicolin   & 4 \\
     FRA10G & 10q11.2  & Common, aphidicolin   & 4 \\
     FRA10C & 10q21    & Common, BrdU          & 6 \\
     FRA10D & 10q22.1  & Common, aphidicolin   & 4 \\
     FRA10A & 10q23.3  & Rare, folic acid    & 1 \\
     FRA10B & 10q25.2  & Rare, BrdU          & 3 \\
     FRA10E & 10q25.2  & Common, aphidicolin   & 4 \\
     FRA10F & 10q26.1  & Common, aphidicolin   & 4 \\
     FRA11C & 11p15.1  & Common, aphidicolin   & 4 \\
     FRA11I & 11p15.1  & Rare, distamycin A  & 2b \\
     FRA11D & 11p14.2  & Common, aphidicolin   & 4 \\
     FRA11E & 11p13    & Common, aphidicolin   & 4 \\
     FRA11H & 11q13    & Common, aphidicolin   & 4 \\
     FRA11A & 11q13.3  & Rare, folic acid    & 1 \\
     FRA11F & 11q14.2  & Common, aphidicolin   & 4 \\
     FRA11B & 11q23.3  & Rare, folic acid    & 1 \\
     FRA11G & 11q23.3  & Common, aphidicolin   & 4 \\
     FRA12A & 12q13.1  & Rare, folic acid    & 1 \\
     FRA12B & 12q21.3  & Common, aphidicolin   & 4 \\
     FRA12E & 12q24    & Common, aphidicolin   & 4 \\
     FRA12D & 12q24.13 & Rare, folic acid    & 1 \\
     FRA12C & 12q24.2  & Rare, BrdU          & 3 \\
     FRA13A & 13q13.2  & Common, aphidicolin   & 4 \\
     FRA13B & 13q21    & Common, BrdU          & 6 \\
     FRA13C & 13q21.2  & Common, aphidicolin   & 4 \\
     FRA13D & 13q32    & Common, aphidicolin   & 4 \\
     FRA14B & 14q23    & Common, aphidicolin   & 4 \\
     FRA14C & 14q24.1  & Common, aphidicolin   & 4 \\
     FRA15A & 15q22    & Common, aphidicolin   & 4 \\
     FRA16A & 16p13.11 & Rare, folic acid    & 1 \\
     FRA16E & 16p12.1  & Rare, distamycin A  & 2b \\
     FRA16B & 16q22.1  & Rare, distamycin A  & 2a \\
     FRA16C & 16q22.1  & Common, aphidicolin   & 4 \\
     FRA16D & 16q23.2  & Common, aphidicolin   & 4 \\
     FRA17A & 17p12    & Rare, distamycin A  & 2a \\
     FRA17B & 17q23.1  & Common, aphidicolin   & 4 \\
     FRA18A & 18q12.2  & Common, aphidicolin   & 4 \\
     FRA18B & 18q21.3  & Common, aphidicolin   & 4 \\
     FRA19B & 19p13    & Rare, folic acid    & 1 \\
     FRA19A & 19q13    & Common, 5-azacytidine & 5 \\
     FRA20B & 20p12.2  & Common, aphidicolin   & 4 \\
     FRA20A & 20p11.23 & Rare, folic acid    & 1 \\
     FRA22B & 22q12.2  & Common, aphidicolin   & 4 \\
     FRA22A & 22q13.1  & Rare, folic acid    & 1 \\
     FRAXB  & Xp22.31  & Common, aphidicolin   & 4 \\
     FRAXC  & Xq22.1   & Common, aphidicolin   & 4 \\
     FRAXD  & Xq27.2   & Common, aphidicolin   & 4 \\
     FRAXA  & Xq27.3   & Rare, folic acid    & 1 \\
     FRAXE  & Xq28     & Rare, folic acid    & 1 \\
     FRAXF  & Xq28     & Rare, folic acid    & 1 \\
  \end{longtable}
\end{center}

%
% BIBLIOGRAPHY
%

\bibliography{thesis}{}
\bibliographystyle{plain}

\end{document}
