% appendix.tex
\chapter{Public data sets analyzed}

\begin{table}
  \centering
  \begin{tabular}{lccr}
    \hline
    Data Set & Figure & Accession & Reference \\ \hline
    \hline
  \end{tabular}
  \caption{Public Data sets Analyzed}
\end{table}


\chapter{Data Collection}

Methylation data was downloaded from the Salk Institute for Biology Studies in
two files from two biological replicates.  Each file contained $\sim600$ million
reads from the

\begin{table}
  \centering
  \begin{tabular}{lccr}
    \hline
    Replicate & Hg18 Reads & Hg19 Reads & Unlifted Reads \\ \hline
    1 & 563,354,527 & 563,071,323 & 566,408 \\
    2 & 620,520,572 & 620,227,842 & 585,460 \\
    \hline
  \end{tabular}
  \caption{Genomic methylation data for IMR90}
\end{table}

\chapter{Data Migration to Human Genome Build 19}

To make valid comparisons between disparate data sets, it is crucial to ensure all data sets are aligned to the same
build of the human genome.  A genome build is a haploid assembly of sequences from several individuals published by
the NCBI to provide a reference for an organism gene and feature set, though not necessarily every allele.  A build assembly
refers to a particular published sequence annotation set.  Human Genome Build 19 (HG19) is the University of California Santa
Cruz nomenclature for the NCBI Build GRCh37 published in 2009\cite{2001Initial}.

In this thesis, the methylation and histone assays were all reported against human genome build 18.  Experiments aligned
against previous builds of the human genome may be updated informatically, either by re-alignment of the
sequence probes or through coordinate transposition.  The UCSC Genome Browser provides a command line utility liftOver for
batch coordinate conversion.  Each file was unzipped and updated to HG19 prior to further analysis using the liftOver tool.

\begin{table}
  \centering
  \begin{tabular}{lccr}
    \hline
    Sample & HG18 Reads & HG19 Reads & Unlifted Reads \\ \hline
    H3K27ac & 16374518 & 16371125 & 3,393 \\
    p65 & 16371125 & 6165230 & 947 \\
    H3K4me1 & 18713234 & 18709033 & 4201 \\
    H3K36me3 & 15808706 & 15807726 & 980 \\
    CTCF & 5501307 & 5499946 & 1361 \\
    \hline
  \end{tabular}
  \caption{Genomic methylation data for IMR90}
\end{table}

\chapter{Iterative Alignment of Probes}

Probes were aligned to the human genome build hg19 using procedures outlined by
Imakaev and colleagues\cite{imakaev2012}.  The chimeric nature of the reads
requires that probes be aligned iteratively, starting from a small, truncated
region from the beginning of the read, mapping this truncated area, increasing
the truncation size and recursing a fixed number of steps or until the alignment
scores become sufficiently poor.  Due to the large number of reads requiring
alignment, we opted to use a fixed truncation length (based on sequence length)
and four steps in the iterative alignment protocol.  The calculation
for the truncation and step size can be found in the the iterativeMapping.py
script provided in the Appendix: Code.  Most reads were 100 base pairs, resulting
in an initial truncation length of 28 base pairs, and step size of 18 base pairs.

Using the mapping functionality from the hiclib python package\cite{imakaev2012},
sequences from the six experimental replicates were realigned to the genome.  The
alignment employed the fast Bowtie2 alignment algorithm\cite{langmead2012}.  Once
aligned, the probes were stored as an interaction matrix in the high performance
HDF5\cite{hdf5} data format, a total of 25Gb for all replicates.

Statistics for iterative alignment are given below:

\begin{center}
  \begin{table}
    \begin{tabular}{l l}
    Total Reads & 2,124,453,478 \\
    Total DS Reads & 1,422,870,270 \\
    Valid Pairs & 713,897,554 \\
    Filtered Reads & 457,298,174 \\
    Percent \textit{trans} Reads & 49.42\% \\
    \end{tabular}
  \end{table}
\end{center}


\chapter{Data Validation}

In order to make meaningful comparisons between data sets (replicates,
in this case), we must show that some degree of relationship exists between
the data sets and comparisons or combinations of the data from disparate sets
are valid to a degree of uncertainty.  It is also essential to understand if the
experimental replicates indeed managed to replicate the conditions of the primary
experiment, or if experimental errors prevent the comparison between replicates.

Spearman's Rank Correlation Coefficient (denoted by the Greek letter $\rho$) is
a non-parametric measure of association between two variables.
Spearman's coefficient assumes some monotonic relationship between variables,
rather than a linear relationship (as in Pearson's), making it appropriate
to compare the IMR90 interaction data sets.  The formula for Spearman's $\rho$ is
given as follows:

\begin{equation}
\rho = 1 - \frac{\sum_{i=1}{n}(d_i^2)}{n(n^2 - 1)}
\end{equation}

where $\rho$ is the correlation coefficient taking values between $-1$ and $+1$,
$d_i = x_i - y_i$ where $x_i, y_i$ are ranks derived from the raw scores $X$ and
$Y$ respectively.

The first replicate IMR90 interaction data set was labeled the primary data set
and the remaining five were compared using Spearman's Rank Correlation.  The
results are given in Table X.

\begin{table}
  \begin{tabular}{|c|*{6}{c|}}
    \toprule
    \textbf{R1} & 0.83 & 0.77 & 0.77 & 0.75 & 0.71 \\ \midrule
    0.83 & \textbf{R2} & 0.82 & 0.83 & 0.79 & 0.75 \\ \midrule
    0.77 & 0.82 & \textbf{R3} & 0.77 & 0.74 & 0.73 \\ \midrule
    0.77 & 0.83 & 0.77 & \textbf{R4} & 0.75 & 0.71 \\ \midrule
    0.75 & 0.79 & 0.74 & 0.74 & \textbf{R5} & 0.69 \\ \midrule
    0.71 & 0.75 & 0.73 & 0.71 & 0.69 & \textbf{R6} \\ \midrule
  \end{tabular}
  \caption{Spearman's $\rho$ across all data sets.}
\label{tab:correlations}
\end{table}

%\chapter{Common Fragile Sites}
%Reproduced from Sutherland, 2012\cite{sutherland2001}.

%\tiny
%\centering
%\caption[Human Fragile Sites]{Human Fragile Sites}
%\label{grid_hfs}
%
%\begin{longtabu} to \textwidth {XXXX}
%  \toprule
%  \rowfont\bfseries Gene Symbol & Chromosome Band Location & Type & Group \\ \bottomrule
%  \endhead%
%
%  \\ \midrule
%  \multicolumn{4}{c}{Continued on next page\ldots}\\
%  \endfoot%
%
%  \\ \bottomrule
%  \endlastfoot%
%
%  FRA1A  & 1p36     & Common, aphidicolin   & 4 \\
%  FRA1B  & 1p32     & Common, aphidicolin   & 4 \\
%  FRA1C  & 2p31.2   & Common, aphidicolin   & 4 \\
%  FRA1L  & 1p31     & Common, aphidicolin   & 4 \\
%  FRA1D  & 1p22     & Common, aphidicolin   & 4 \\
%  FRA1M  & p21.3    & Rare, folic acid    & 1 \\
%  FRA1E  & 1p21.2   & Common, aphidicolin   & 4 \\
%  FRA1J  & 1q12     & Common, 5-azacytidine & 5 \\
%  FRA1F  & 1q21     & Common, aphidicolin   & 4 \\
%  FRA1G  & 1q25.1   & Common, aphidicolin   & 4 \\
%  FRA1K  & 1q31     & Common, aphidicolin   & 4 \\
%  FRA1H  & 1q42     & Common, 5-azacytidine & 5 \\
%  GRA1I  & 1q44     & Common, aphidicolin   & 4 \\
%  FRA2C  & 2p24.2   & Common, aphidicolin   & 4 \\
%  FRA2D  & 2p16.2   & Common, aphidicolin   & 4 \\
%  FRA2E  & 2p13     & Common, aphidicolin   & 4 \\
%  FRA2L  & 2p11.2   & Rare, folic acid    & 1 \\
%  FRA2A  & 2q11.2   & Rare, folic acid    & 1 \\
%  FRA2B  & 2q13     & Rare, folic acid    & 1 \\
%  FRA2F  & 2q21.3   & Common, aphidicolin   & 4 \\
%  FRA2K  & 2q22.3   & Rare, folic acid    & 1 \\
%  FRA2G  & 2q31     & Common, aphidicolin   & 4 \\
%  FRA2H  & 2q32.1   & Common, aphidicolin   & 4 \\
%  FRA2I  & 2q33     & Common, aphidicolin   & 4 \\
%  FRA2J  & 2q37.3   & Common, aphidicolin   & 4 \\
%  FRA3A  & 3p24.2   & Common, aphidicolin   & 4 \\
%  FRA3B  & 3p14.2   & Common, aphidicolin   & 4 \\
%  FRA3D  & 3q25     & Common, aphidicolin   & 4 \\
%  FRA3C  & 3q27     & Common, aphidicolin   & 4 \\
%  FRA4A  & 4p16.1   & Common, aphidicolin   & 4 \\
%  FRA4D  & 4p15     & Common, aphidicolin   & 4 \\
%  FRA4B  & 4q12     & Common, BrdU          & 6 \\
%  FRA4C  & 4q31.1   & Common, aphidicolin   & 4 \\
%  FRA5E  & 5p14     & Common, aphidicolin   & 4 \\
%  FRA5A  & 5p13     & Common, BrdU          & 6 \\
%  FRA5B  & 5q15     & Common, BrdU          & 6 \\
%  FRA5D  & 5q15     & Common, aphidicolin   & 4 \\
%  FRA5F  & 5q21     & Common, aphidicolin   & 4 \\
%  FRA5C  & 5q31.1   & Common, aphidicolin   & 4 \\
%  FRA5G  & 5q35     & Rare, folic acid    & 1 \\
%  FRA6B  & 6p25.1   & Common, aphidicolin   & 4 \\
%  FRA6A  & 6p23     & Rare, folic acid    & 1 \\
%  FRA6C  & 6p22.2   & Common, aphidicolin   & 4 \\
%  FRA6D  & 6q13     & Common, BrdU          & 6 \\
%  FRA6G  & 6q15     & Common, aphidicolin   & 4 \\
%  FRA6F  & 6q21     & Common, aphidicolin   & 4 \\
%  FRA6E  & 6q26     & Common, aphidicolin   & 4 \\
%  FRA7B  & 7p22     & Common, aphidicolin   & 4 \\
%  FRA7C  & 7p14.2   & Common, aphidicolin   & 4 \\
%  FRA7D  & 7p13     & Common, aphidicolin   & 4 \\
%  FRA7A  & 7p11.2   & Rare, folic acid    & 1 \\
%  FRA7J  & 7q11     & Common, aphidicolin   & 4 \\
%  FRA7E  & 7q21.2   & Common, aphidicolin   & 4 \\
%  FRA7F  & 7q22     & Common, aphidicolin   & 4 \\
%  FRA7G  & 7q31.2   & Common, aphidicolin   & 4 \\
%  FRA7H  & 7q32.3   & Common, aphidicolin   & 4 \\
%  FRA7I  & 7q36     & Common, aphidicolin   & 4 \\
%  FRA8B  & 8q22.1   & Common, aphidicolin   & 4 \\
%  FRA8A  & 8q22.3   & Rare, folic acid    & 1 \\
%  FRA8C  & 8q24.1   & Common, aphidicolin   & 4 \\
%  FRA8E  & 8q24.1   & Rare, distamycin A  & 2b \\
%  FRA8D  & 8q24.3   & Common, aphidicolin   & 4 \\
%  FRA9A  & 9p21     & Rare, folic acid    & 1 \\
%  FRA9C  & 9p21     & Common, BrdU          & 6 \\
%  FRA9F  & 9q12     & Common, 5-azacytidine & 5 \\
%  FRA9D  & 9q22.1   & Common, aphidicolin   & 4 \\
%  FRA9B  & 9q32     & Rare, folic acid    & 1 \\
%  FRA9E  & 9q32     & Common, aphidicolin   & 4 \\
%  FRA10G & 10q11.2  & Common, aphidicolin   & 4 \\
%  FRA10C & 10q21    & Common, BrdU          & 6 \\
%  FRA10D & 10q22.1  & Common, aphidicolin   & 4 \\
%  FRA10A & 10q23.3  & Rare, folic acid    & 1 \\
%  FRA10B & 10q25.2  & Rare, BrdU          & 3 \\
%  FRA10E & 10q25.2  & Common, aphidicolin   & 4 \\
%  FRA10F & 10q26.1  & Common, aphidicolin   & 4 \\
%  FRA11C & 11p15.1  & Common, aphidicolin   & 4 \\
%  FRA11I & 11p15.1  & Rare, distamycin A  & 2b \\
%  FRA11D & 11p14.2  & Common, aphidicolin   & 4 \\
%  FRA11E & 11p13    & Common, aphidicolin   & 4 \\
%  FRA11H & 11q13    & Common, aphidicolin   & 4 \\
%  FRA11A & 11q13.3  & Rare, folic acid    & 1 \\
%  FRA11F & 11q14.2  & Common, aphidicolin   & 4 \\
%  FRA11B & 11q23.3  & Rare, folic acid    & 1 \\
%  FRA11G & 11q23.3  & Common, aphidicolin   & 4 \\
%  FRA12A & 12q13.1  & Rare, folic acid    & 1 \\
%  FRA12B & 12q21.3  & Common, aphidicolin   & 4 \\
%  FRA12E & 12q24    & Common, aphidicolin   & 4 \\
%  FRA12D & 12q24.13 & Rare, folic acid    & 1 \\
%  FRA12C & 12q24.2  & Rare, BrdU          & 3 \\
%  FRA13A & 13q13.2  & Common, aphidicolin   & 4 \\
%  FRA13B & 13q21    & Common, BrdU          & 6 \\
%  FRA13C & 13q21.2  & Common, aphidicolin   & 4 \\
%  FRA13D & 13q32    & Common, aphidicolin   & 4 \\
%  FRA14B & 14q23    & Common, aphidicolin   & 4 \\
%  FRA14C & 14q24.1  & Common, aphidicolin   & 4 \\
%  FRA15A & 15q22    & Common, aphidicolin   & 4 \\
%  FRA16A & 16p13.11 & Rare, folic acid    & 1 \\
%  FRA16E & 16p12.1  & Rare, distamycin A  & 2b \\
%  FRA16B & 16q22.1  & Rare, distamycin A  & 2a \\
%  FRA16C & 16q22.1  & Common, aphidicolin   & 4 \\
%  FRA16D & 16q23.2  & Common, aphidicolin   & 4 \\
%  FRA17A & 17p12    & Rare, distamycin A  & 2a \\
%  FRA17B & 17q23.1  & Common, aphidicolin   & 4 \\
%  FRA18A & 18q12.2  & Common, aphidicolin   & 4 \\
%  FRA18B & 18q21.3  & Common, aphidicolin   & 4 \\
%  FRA19B & 19p13    & Rare, folic acid    & 1 \\
%  FRA19A & 19q13    & Common, 5-azacytidine & 5 \\
%  FRA20B & 20p12.2  & Common, aphidicolin   & 4 \\
%  FRA20A & 20p11.23 & Rare, folic acid    & 1 \\
%  FRA22B & 22q12.2  & Common, aphidicolin   & 4 \\
%  FRA22A & 22q13.1  & Rare, folic acid    & 1 \\
%  FRAXB  & Xp22.31  & Common, aphidicolin   & 4 \\
%  FRAXC  & Xq22.1   & Common, aphidicolin   & 4 \\
%  FRAXD  & Xq27.2   & Common, aphidicolin   & 4 \\
%  FRAXA  & Xq27.3   & Rare, folic acid    & 1 \\
%  FRAXE  & Xq28     & Rare, folic acid    & 1 \\
%  FRAXF  & Xq28     & Rare, folic acid    & 1 \\
%\end{longtabu}

