% Acronym Definitions
\newacronym{DNA}{DNA}{Deoxyribonucleic Acid}
\newacronym{RNA}{RNA}{Ribonucleic Acid}
\newacronym{GEO}{GEO}{Gene Expression Omnibus}
\newacronym{NCBI}{NCBI}{National Center for Biotechnology Information}
\newacronym{SRA}{SRA}{Sequence Read Archive}
\newacronym{TAD}{TAD}{Topologically Associating Domain}
\newacronym{LAD}{LAD}{Lamina Associating Domain}
\newacronym{PCA}{PCA}{Principal Component Analysis}
\newacronym{SVD}{SVD}{Singular Value Decomposition}
\newacronym{PC}{PC}{Principal Component}

% Glossary definitions
\newglossaryentry{karyotype}{%
  name={karyotype},
  description={A photomicrograph of chromosomes arranged according to a standard classification.}%
}

\newglossaryentry{polymer}{%
  name={polymer},
  description={A substance that has a molecular structure consisting chiefly or entirely of a large number of similar units bonded together, e.g., many synthetic organic materials used as plastics and resins.}
}

\newglossaryentry{epigenetic}{%
  name={epigenetic},
  description={Any heritable influence on gene activity, unaccompanied by a change in the DNA}
}

\newglossaryentry{restriction enzyme}{%
  name={restriction enzyme},
  description={An enzyme that restricts, or performs double-strand cut, at a specific DNA sequence motif.}
}

\newglossaryentry{ligation}{%
  name={ligation},
  description={The joining of two DNA strands or other molecules by a phosphate ester linkage.}
}

\newglossaryentry{oligonucleotide}{%
  name={oligonucleotide},
  description={A polynucleotide whose molecules contain a relatively small number of nucleotides.}
}

\newglossaryentry{nucleosome}{%
  name={nucleosome},
  description={A chromatin secondary structure consisting of $\sim147$ base pairs of DNA wrapped 1.75 times around an octamer of core histone proteins.}
}

\newglossaryentry{scree plot}{%
  name={scree plot},
  description={A graph displaying the eigenvalue associated with each component in descending order versus the component number.  Used in PCA to determine how many components to consider in the analysis.}
}

\newglossaryentry{eigenspectrum}{%
  name={eigenspectrum},
  description={A spectrum of eigenvalues.}
}
