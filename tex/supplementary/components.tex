\newpage
\section*{Principal Components}

We calculated Spearman's correlation coefficients between principle components for each data set.

% data from  pca/correlations/*.csv
\begin{table}[h]
  \centering
  \caption{Correlation of PC1 between all experimental replicates}\label{table:PC1Correlations}
  \begin{tabularx}{\textwidth}{@{}cccccccc@{}}
    \toprule
    \multicolumn{6}{l}{IMR90} & \multicolumn{2}{l}{hESC} \\
    \midrule
    \textbf{R1} & 0.981       & 0.972       & 0.983       & 0.983       & 0.977       & 0.656       & 0.658 \\
    0.981       & \textbf{R2} & 0.985       & 0.982       & 0.976       & 0.974       & 0.654       & 0.652 \\
    0.972       & 0.985       & \textbf{R3} & 0.982       & 0.977       & 0.980       & 0.634       & 0.629 \\
    0.983       & 0.982       & 0.982       & \textbf{R4} & 0.986       & 0.982       & 0.655       & 0.655 \\
    0.983       & 0.976       & 0.977       & 0.986       & \textbf{R5} & 0.98        & 0.633       & 0.633 \\
    0.977       & 0.974       & 0.980       & 0.982       & 0.985       & \textbf{R6} & 0.610       & 0.607 \\
    0.656       & 0.654       & 0.634       & 0.655       & 0.633       & 0.610       & \textbf{R1} & 0.973  \\
    0.658       & 0.652       & 0.629       & 0.655       & 0.633       & 0.607       & 0.973       & \textbf{R2}    \\
    \bottomrule
  \end{tabularx}
\end{table}

We also examined the scree plots of each \gls{PC} by experiment and replicate.  The plot for hESC R1 is included for reference.

\begin{figure}
  \centering
  \includegraphics[width=\textwidth]{./figures/supplementary/hESC-R1-scree.png}\label{fig:hESCScree}
\end{figure}
