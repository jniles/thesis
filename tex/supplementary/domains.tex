\newpage
\section*{Domain Discovery}\label{sec:SuppDomainDiscovery}

Using the directionality indices computed on at 10kb resolution map, we computed domains by selecting the regions with the top
$10\%$ directionality bias.  Domains were further refined by intersecting between replicates and discarding non-overlapping
domains, leaving a small, conserved subset.  Overlaps between window sizes and replicates are shown in the figures below.

\subsection*{Domain Conservation}\label{sec:domainConservation}

\begin{figure}[H]
  \caption{Conserved Domains between IMR90 Replicates}
  \includegraphics[width=\textwidth]{figures/supplementary/domains/venn3.png}\label{fig:SuppVenn3}
  \small
  Domain conservation between the first three IMR90 replicates.  The window size here is 200kb for each
  replicate.
\end{figure}

\begin{figure}[H]
  \caption{Domain Size Histograms}
  \begin{minipage}{0.5\textwidth}%
    \includegraphics[width=\textwidth]{./figures/supplementary/domains/size200kb.png}
  \end{minipage}%
  \hfill
  \begin{minipage}{0.5\textwidth}
    \includegraphics[width=\textwidth]{./figures/supplementary/domains/size400kb.png}
  \end{minipage}%

  \vfill

  \begin{minipage}{0.5\textwidth}%
    \includegraphics[width=\textwidth]{./figures/supplementary/domains/size800kb.png}
  \end{minipage}%
  \hfill
  \begin{minipage}{0.5\textwidth}
    \includegraphics[width=\textwidth]{./figures/supplementary/domains/size1000kb.png}
  \end{minipage}
  \small
  Histogram of domain sizes computed for IMR90 Replicate V.
\end{figure}

\begin{figure}[H]
  \caption{Domain Overlaps by Window Size}
  \begin{minipage}{0.5\textwidth}%
    \includegraphics[width=\textwidth]{./figures/supplementary/domains/venn2-IMR90-R1-100-vs-IMR90-R1-200.png}
  \end{minipage}%
  \hfill
  \begin{minipage}{0.5\textwidth}
    \includegraphics[width=\textwidth]{./figures/supplementary/domains/venn2-IMR90-R1-200-vs-IMR90-R1-400.png}
  \end{minipage}%

  \vfill

  \begin{minipage}{0.5\textwidth}%
    \includegraphics[width=\textwidth]{./figures/supplementary/domains/venn2-IMR90-R1-400-vs-IMR90-R1-800.png}
  \end{minipage}%
  \hfill
  \begin{minipage}{0.5\textwidth}
    \includegraphics[width=\textwidth]{./figures/supplementary/domains/venn2-IMR90-R1-800-vs-IMR90-R1-1000.png}
  \end{minipage}
  \small
  Venn diagrams illustrating overlaps between domains discovered at increasing window sizes.  In general, the
  increasing window size led to increasingly larger domains.  Larger domains often subsumed domains discovered
  with a smaller window size.
\end{figure}

\newpage
\subsection*{Domain Overlaps with Lesions}

\begin{figure}[H]
  \caption{Lesion Overlaps by Window Size}
  \begin{minipage}{0.5\textwidth}%
    \includegraphics[width=\textwidth]{./figures/supplementary/domains/boundaries-IMR90-200kb.png}
  \end{minipage}%
  \hfill
  \begin{minipage}{0.5\textwidth}
    \includegraphics[width=\textwidth]{./figures/supplementary/domains/boundaries-IMR90-400kb.png}
  \end{minipage}%

  \vfill

  \begin{minipage}{0.5\textwidth}%
    \includegraphics[width=\textwidth]{./figures/supplementary/domains/IMR90boundaries200kbwindows10000kbslop.png}
  \end{minipage}%
  \hfill
  \begin{minipage}{0.5\textwidth}
    \includegraphics[width=\textwidth]{./figures/supplementary/domains/IMR90boundaries400kbwindows10000kbslop.png}
  \end{minipage}
  \small
  Lesions are clustered around domain boundaries at higher frequencies than expected by chance.  Using shuffling and
  resampling procedures (See Section~\ref{sec:resampling}), we constructed the null hypothesis `lesions are distributed evenly
  throughout boundaries and non-boundary regions.'  In regions of 5kb (top) and 10kb (bottom) around the boundary regions, we are
  able to reject the null hypothesis with negligible $p-$value.
\end{figure}

\begin{figure}[H]
  \caption{Mutation Types from TCGA}
  \includegraphics[width=\textwidth]{./figures/supplementary/domains/mutationTypePieChart.png}
\end{figure}

