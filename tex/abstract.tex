\begin{abstract}
Cancer is an enigmatic disease of dysregulation, estimated to cause over half a million deaths in the United States in 2015 \citep{siegel2015}.
A patient's prognosis is heavily dependent on the time of detection, fueling research into the origins, biomarkers, and early detection methods for all types of cancers \citep{hirsch2001}.
We hypothesized that the frequency of cancerous lesions in lung cancer cells may be a function of morphological features in genome architecture.  Using chromatin conformation
maps of human embryonic stem cells and lung cancer cells, we reproduce previously discovered topological domains, and show that cancerous lesions occur preferentially along
domain boundaries. Furthermore, we find that topological domains are conserved and form a natural ordering of chromatin at different genomic scales.  These findings strengthen
the notion that topological domains are a fundamentally important level of genomic regulation, and motivate research into the potential relationship between topological
structure and disease.
\end{abstract}
