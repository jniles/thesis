\begin{abstract}
Cancer is an enigmatic disease of dysregulation, estimated to cause over half a million deaths in 2015 according to the American Cancer Society\cite{acs2015}.
A patient's prognosis is heavily dependent on the time of detection, fueling research into the origins, biomarkers, and early detection methods for all types of cancers\cite{hirsch2001}.
We hypothesized that the frequency of cancerous lesions in lung cancer cells may be function of morphological changes in genome architecture.  Using human embryonic
stem cells and lung cancer cells, we show that previously discovered topological domains are cell type dependent, and show significant correlation with mutational
frequencies in cancer.  These findings provide motivation for research into the relationships between the dynamic structure of the genome and the propensity to
develop mutations leading to deadly cancers.
\end{abstract}
