% discussion.tex
\chapter{Discussion}

\section*{Normalization and Interaction Scaling}
Notably, the scaling ratios are unchanged by normalization, indicating that scaling may be a property of the underlying distribution
rather than an artifact of out data processing.  Indeed, this indicates that perhaps normalization should be performed, as is
done in HiCNorm \citep{hu2012} on a chromosome by chromosome basis, estimating different background distributions for each chromosome
rather than using a generalized fitting algorithm such as IPF\@.

\section*{Chromatin Compartmentalization}

The original Hi-C experiment revealed that the genome can be compartmentalized by \gls{PC} into two characteristic classes, arbitrarily
labeled A and B.  Lieberman-Aiden and colleagues showed positive correlations between compartment identity, gene density, and chromatin
accessibility, concluding that compartment A consists of largely open chromatin, while compartment B is densely packed \citep{aiden2009}.
Interestingly, we did not observe any correlation between the eigenvector and gene expression data (via genome-wide mRNA expression,
Spearman's $\rho = -0.014$, p negligible; Supplementary Information~\ref{}).  Furthermore, shifts in compartment character did not correlate to
changes in gene expression level (Spearman's $\rho = -0.01$, p negligible).  Given these results, it seems unlikely that cells regulate
gene expression at the compartment level.  Compartmentalization appears to be part of the larger nuclear architectural scheme, rather than
a dynamic regulatory mechanism.
