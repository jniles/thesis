% discussion.tex
\chapter{Discussion}

The shadowy mechanisms that bring about catastrophic diseases such as cancer are slowly being elucidated.  Using previously published normalization
algorithms, we were able to reproduce topological domain and chromatin compartments seen in previous papers \citep{dekker2012, dixon2012}. Our results
yielded two intriguing findings: chromatin compartments do not seem to serve a regulatory role, and cancerous lesions are clustered within domains
and domain boundaries.

\section*{Chromatin Compartments and Regulation}

The original Hi-C experiment revealed that the genome can be compartmentalized by \gls{PC} into two characteristic classes, arbitrarily
labeled A and B.  Lieberman-Aiden and colleagues showed positive correlations between compartment identity, gene density, and chromatin
accessibility, concluding that compartment A consists of largely open chromatin, while compartment B is densely packed \citep{aiden2009}.
Interestingly, we did not observe any correlation between the eigenvector and gene expression data (via genome-wide mRNA expression,
Spearman's $\rho = -0.014$, $p-$value negligible; Figure~\ref{fig:expressionChangeByCompartment}).  Furthermore, shifts in compartment character did not correlate to
changes in gene expression level (Spearman's $\rho = -0.01$, $p-$value negligible).  Given these results, it seems unlikely that cells regulate
gene expression at the compartment level.

\section*{Lesions and Topological Domains}

The pursuit of an underlying topological explanation for genomic fragility is not yet over.  We demonstrated that boundary regions of topological
domains show higher numbers of mutations than expected on average, yet the mechanistic insight into the causes of these mutations is yet to be
fully described.  Early results show differentiable peaks for certain \gls{DNA} binding proteins; however, a full set of epigenetic factors (histone
modifications, transcription factors, etc) should be analyzed for a cohesive understanding of their importance.  The origin of genomic fragility on
the macroscopic level defined by common fragile sites is yet undiscovered, and we did not observe a relationship between the macro-molecular
compartments seen by \citet{dekker2012} and fragile site placement.  It may be that fragile sites are organized on higher orders, such as chromosome
territories.

Many questions remain: what types of lesions occur most frequently at domain boundaries?  How are the genome's fractally structured topological
domains regulated?  In this analysis, we studied lung cancer specific mutations; however, there is strong evidence that boundaries are conserved
across cell types \citep{dixon2012,pope2014}.  Future studies can expand off this base, searching for mutations across many different cancers to
identify a conserved mechanism.  Finally, there is still much work to be done in characterizing the role that topological domains play in the
greater scheme of genome architecture and expression.  Only once we understand the intricate interactions surrounding these domains, will we be
equipped conquer the diseases that plague them.
