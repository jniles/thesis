% introduction.tex

\chapter{Introduction}

What causes the genome to break?  The human genome is an assemblage of six billion nucleotides encoding the vast instruction set
which controls cell growth, function, and death.  Every possible gene expression pattern is embedded in the genome, yet particular
cell and tissue types arise when the instructions are compiled differently during differentiation.  Some of the great mysteries
of molecular biology surround the establishment and maintenance of the genome structure, and its implications on cell fate. 
Current research indicates the establishment of this \gls{epigenetic} landscape determining cell fate may play an important
role in cancer genesis.

The emerging view of genomic organization indicates that cell regulate expression in local, topologically distinct domains
\citep{guelen2008,dixon2012,pope2014}.  These domains, called topological domains, present a new regulatory agent not
previously described.  Recent work analyzed the architecture \citep{imakaev2012} and epigenetic characteristics
\citep{dixon2012, pope2014}, with implications on gene expression regulation and replication timing.

Topological domains present a novel lens through which to investigate disease, particularly diseases with profound architectural components.
We investigated the structural changes during cell differentiation to establish an architectural link between
nuclear topology and the probability of developing lesions or breaks in the genomic sequence.  We hypothesized that mutations
occurring frequently in specific cancers are based on the epigenetic architecture of the original cell type.  Using human
embryonic stem cells and lung fibroblasts as models, we proposed that topological changes similar to those that establish
patterns of differentiation are responsible for introducing lesions seen in many cell type specific cancers.  To validate our
claim, we applied a heuristic to discover local chromatin topological domains in lung fibroblasts and aligned these domains and
their boundaries to lesions found in the \gls{TCGA}.

This document will proceed as follows.  First, the analytical tools used to perform analysis on genomic data sets are introduced.
These tools include iterative normalization of chromatin contact maps, principal component analysis, and an algorithm to detect
topological domains from contact maps.  A literature review of chromatin architecture is presented to
provide a strong biological intuition regarding chromatin conformation.  Our methods for data acquisition and processing is
described, and we conclude with a discussion of the results and propose areas of further investigation and improvement.
