% introduction.tex

\chapter{Introduction}

What causes the genome to break?  The human genome is a six billion nucleotide string that encodes the
the vast instruction set for the cells growth, function, and death.  All possible expression patterns are
encoded in this instruction set, yet particular cell types arise when the instructions are compiled differently
during differentiation.  Modern cancer research indicates the establishment of this epigenetic landscape
determining cell fate may play a strategic role in cancer genesis.

We investigate the mechanical and structural changes during differentiation in order to establish an
architectural link between nuclear topology and the probability of developing cancerous lesions.  We
assume that mutations occurring frequently in specific cancers are based on the epigenetic architecture
of the original cell type.  Using human embryonic stem cells and lung fibroblasts, we propose that
topological changes similar to those that establish patterns of differentiation are responsible for
introducing lesions seen in many cell type specific cancers.

This thesis will proceed as follows.  First, the analytical tools to perform data analysis on genomic
data sets is introduced.  These tools include iterative normalization of chromatin contact maps, eigenvalue
decomposition, and an algorithm to detect topologically associating domains from normalized contact maps.
A literature review of chromatin architecture is presented to provide a strong biological foundation from which
to interpret results.  The methods for data acquisition and processing is described, and we conclude with a
discussion of the results and propose areas of further investigation and improvement.
