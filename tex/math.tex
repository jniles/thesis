% math.tex

\chapter{Analytical Preliminaries}

This thesis endeavors to secure a relationship between chromatin structure and fragility that may potentially lead to
disease states.  The analysis will include gene expression profiles, iterative curve fitting, and principal component analysis.  This
beginning will serve as an introduction to the numerical methods used to compare and contrast gene expression and chromatin states
between the two cell lines examined within.

\section{Normalization of Genomic Contact Maps}

The Hi-C experiment generates a genomic contact map describing the number of observed contacts between any two genomic
regions.  All \textit{double stranded} reads (those which map uniquely to two genomic loci) can be arranged in a square
matrix by binning the genome into evenly spaced bins.  These bins are arranged into an $n \times n$ matrix $O$ in which the
cell $i,j$ represents the number of observed contacts between bins $i$ and $j$.  Depending on the data quality, these bins
may be as small as 10 kilobases or as large as 2 megabases.

In data analysis, the first step is always to normalize the data set to remove experimental biases and establish reproducibiility.
Several methods have been proposed to normalize Hi-C interaction matrices.

% TODO


\section{Expression Profiling}

The set of genes expressed in a cell, collectively known as the \textit{transcriptome}, provides insight into chromatin accessibility
and cell state.  A gene expressed in $n$ conditions (or $n$ samples) forms vector in $n-$dimensional space.  Similarly, the expression
profile the entire transcriptome for a single sample is a vector in $m-$dimensional space, where $m$ is the number of genes assessed.
We therefore denote an experiment interrogating $n$ genes across $m$ samples as an $m \times n$ matrix $S_{m,n}$.
\[
  S_{m,n} = \left[
    \begin{array}{cccc}
      s_{1,1} & s_{1,2} & \cdots & s_{1,n} \\
      s_{2,1} & s_{2,2} & \cdots & s_{2,n} \\
      \vdots & \vdots & \ddots  & \vdots \\
      s_{m,1} & s_{m,2} & \cdots & s_{m,n}
    \end{array}
  \right]
\]

and a single gene's expression profile is
\[ G_{i} = \begin{pmatrix} g_{i1} & g_{i2} & g_{i3} & \cdots & g_{in} & \end{pmatrix} \]

We wish to compare differences in gene expression profiles for particular, or all, genes between samples.  This amounts to comparing
\textit{distances} between gene expression vectors.  There are several methods for doing so.

\subsection{Euclidean Distance}

The simplest genomic distance measure is \textbf{Euclidean} distance.  Euclidean distance is a particular \textit{norm}, known as
the 2-norm.  Geometrically, the norm is a measure of `size' or `proximity' in $n-$dimensional space.  The 2-norm for a vector $x$ is
defined as
\[
  \|x\|_{2} = {(|x_1|^2 + |x_2|^2 + \cdots + |x_n|^2)}^{\frac{1}{2}}
\]

The distance between two vectors in a vector space can now be found using the triangle inequality on the 2-norm.  As a norm, the Euclidean
norm has some desirable properties, notably, non-negativity, positivity, and a definition of the triangle inequality
$\|x + y\| = \|x\| + \|y\|$~\cite{horn2013}.  Directly from the triangle inequality, we define the Euclidean distance in $n-$dimensions to
be
\[
  D_{Euc}(x,y) = \|x\| - \|y\| = {(\sum_{i = 1}^{n}{(a_i - b_i)}^2)}^{\frac{1}{2}}
\]

\subsection{Correlations between Vectors}

\textbf{TODO}

\section{Iterative Correction of Interaction Maps}

Massive data mining experiments must be cognizant of any inherent biases in the data set and seek to eliminate them prior to performing
meaningful data analysis.  In this thesis, we depend on accurate reporting of contact frequency between genomic loci, which is prone
to a number of experimental, mechanical, and biological biases\cite{dekker2006}.  In order to compare data sets, I employ an algorithm,
iterative correction and eigenvector decomposition (ICE) developed by Imakaev and colleagues\cite{imakaev2012} established on previous
work by Tanay and Yaffe\cite{yaffe2011}.  In the following section, I will describe the motivation for ICE, the algorithm in detail, and
prove convergence properties for the algorithm.

Genome-wide interaction maps form matrix of observed contact frequencies between genomic segments.  Observed contacts are expected to
be noisy do to biases introduced in the experimental procedure and inherent differences in mappability for different genomic regions
(e.g.\ centromeres).

% Why are biases factorizable?

The ICE algorithm decomposes a contact map into the product of true contact frequencies and a set of biases for every region
\[
  O_{ij} = B_{i}B_{j}T_{ij}
\]
wjth $T_{ij}$ normalized as $\sum_{i,i \neq j,j\pm1}T_{ij} = 1$ for each region $j$.
