\chapter{Methods}

The methods used in our analyses depend on publicly available open-source packages and custom scripts developed in house.
These scripts are permissively licensed under the GNU public license (version 3) and
\href{https://github.com/New-College-of-Florida/Jonathan-Niles-Thesis}{\underline{available online}}.  Hi-C analysis was conducted using
the excellent hiclib from the lab of Leonid Mirny \citep{imakaev2012}.  Visualizations were created in the R programming
language \citep{r2014} and the matplotlib python package \citep{hunter2007}.  Genome distance calculations leveraged the
bedtools suite \citep{quinlan2010} and pybedtools package \citep{dale2011}.

\section*{Data Collection and Processing}

IMR90 lung fibroblast and H1 \glsentryfull{hESC} cell lines were examined due to the availability of data and body of preexisting
literature.  Hi-C data sets for both cell lines were obtained from \glsentryfull{GEO} \citep{edgar2002}, accession number
\href{http://www.ncbi.nlm.nih.gov/geo/query/acc.cgi?acc=GSE43070}{GSE43070}.  Two and six technical replicates were examined for
\gls{hESC} and IMR90 cell lines respectively.  Chimeric HI-C reads were aligned to the human genome build 19 using the BowTie2
alignment software\footnote{\url{http://bowtie-bio.sourceforge.net/bowtie2/index.shtml}} \citep{langmead2012}. Reads containing
segments mapping to unique portions of the genome were stored on disk in the high performance
HDF5\footnote{\url{http://www.hdfgroup.org/HDF5/}} format using the \href{https://bitbucket.org/mirnylab/hiclib}{hiclib} python
library \citep{imakaev2012}.  Low-scoring or single sided reads were discarded from future analysis.  Interaction maps were
stored in 2Mb, 1Mb, 200kb, 40kb, 20kb, and 10kb bin sizes.

In addition to plotting, the average scaling properties were computed as follows.  For each off chromosome, the number average
number of contacts at a given distance was computed by averaging the $k-$diagonal of the contact matrix

\begin{align}
  c_j = \frac{\sum_{i=1}^{n - j} \matr{O_{i, i+k}}}{n - j} \label{eq:probeScale}
\end{align}

The vector $\vec{c} = \left(c_1, c_2, \hdots, c_{n-1}\right)$ records the average number of contacts see across the chromosome
at each distance.

\section*{Iterative Correction and Eigenvector Decomposition}

Exported maps were corrected using the `correctIC' function from hiclib.  The heatmaps were iteratively corrected as previously
described \citep{imakaev2012}, at an average of 10 iterations per map.  Pearson correlations were calculated between each sample
for both replicates.  The maps were decomposed into eigenvector components by \gls{PCA} using the `doPCA' functions in hiclib.
\gls{PCA} was performed on interaction maps of 1Mb and 200kb bin sizes.

\section*{Gene Expression and Pathway Analysis}

Gene expression data sets were obtained from \gls{GEO} (\href{http://www.ncbi.nlm.nih.gov/geo/query/acc.cgi?acc=GSE2672}{GSE2672},
\href{http://www.ncbi.nlm.nih.gov/geo/query/acc.cgi?acc=GSE54186}{GSE54186}) for each cell line \citep{kim2005} \citep{kim2014}.  Probe
annotations for the Affymetrix GeneChip Human Genome U133 Plus 2.0 microarray and gene annotations for hg19 were obtained from
BioMart \citep{kasprzyk2011}.  Upregulated genes were determined by taking a $\log_2$ ratio of the mean signal intensity between
replicates

\begin{align}
  \delta_i = \log_2{\frac{\bar{E_i^I}}{\bar{E_i^H}}}
\end{align}

where $\delta_i$ is the expression change for a gene $i$, $\bar{E_i^I}$ and $\bar{E_i^H}$ are the mean expression values across
IMR90 and hESC replicates, respectively.  Pathway analysis was performed using the online utilities provided by
\href{http://consensuspathdb.org/}{ConsensusPathDB} \citep{kamburov2012}.

\section*{Directionality Index and Domain Discovery}

The directionality index was calculated as previously described by Dixon and colleagues \citep{dixon2012}.  We used the highest
resolution (10kb) map generated by \gls{ICE} to compute the directionality index for 100kb, 200kb, 400kb, 800kb, 1Mb windows
genome wide.  We reasoned that domains exist between indices where a sign change occurs of the directionality index.  We located
regions of the directionality index with the 10\% largest peaks, and defined a topological domain to be the entire positive (negative)
region beneath (above) the index between sign changes.   `True' domains were determined as conserved domains between replicates.
The domains were exported to \.bed format for analysis using the pybedtools package \citep{quinlan2010,dale2011}.

\section*{Comparison to Cancerous Lesions}

Lung cancer data sets were obtained from the \gls{TCGA} for
\href{https://tcga-data.nci.nih.gov/tcga/tcgaCancerDetails.jsp?diseaseType=LUAD&diseaseName=Lung\%20adenocarcinoma}{514 cases}
of lung adenocarcinoma \citep{cerami2012,gao2013}.  The frequency of lung cancer mutations within domains were computed by resampling
a randomized set of domains.  For each domain size, a normal distribution (null hypothesis $H_0$) was constructed by shuffling domains
intrachromosomally $1000$ times, and recording the overlaps between mutations and domains.  $p-values$ were calculated as

\begin{align}
  p-value &= P(X \geq T | H_0)
\end{align}

where $T$ is the test statistic, $X$ is the observed value, and the function $P(\dots)$ is the probability function on $X$.  The upper-tailed
test was used to calculate the $p-value$ for the shuffled distribution ($p-value = 1 - cdf(T)$).  The same test was performed on windows
$\pm5$kb and $\pm10$kb windows surrounding the domain boundaries.

\section*{Bio-marker Discovery}

The epigenetic state surrounding boundaries were evaluated in windows of $\pm500$kb using bedtools.  Regions of $\pm500$kb around
domain boundaries were divided into $100$bp windows, and chromatin modeling proteins MAFK, Rad21, CEBPB, and CTCF were mapped to
the bins, as well as RNA polymerase II\@.  Plots were generated using the R programming language, and lowess smoothing was applied
to the data to form well-defined peaks (smoothing parameter $\frac{1}{12}$).
