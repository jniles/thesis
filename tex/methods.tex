\chapter{Methods}

\section*{Data Collection}

We selected IMR90 lung fibroblast and H1 \glsentryfull{hESC} cell lines for analysis due to the availability of data and
preexisting literature on these cells lines.

\subsection*{Processing Chromatin Interaction Data}

We obtained Hi-C data sets from the Gene Expression Omnibus (GEO)\cite{edgar2002} as \gls{Sequence Read Archive} files,
accession number \href{http://www.ncbi.nlm.nih.gov/geo/query/acc.cgi?acc=GSE43070}{GSE43070}.  Chimeric reads were realigned
to the human genome (hg19) using the BowTie2 alignment algorithm on 16 threads\cite{langmead2012}.  The resultant mapped
reads were stored as HDF5 heatmaps.

\subsection*{Iterative correction}
Using the Hi-C library developed by Imakaev and colleagues\cite{imakaev2012}, the heatmaps were iteratively corrected
and decomposed into principal components.

% FIXME Make sure appendix numbers/labels are correct
We noted that the aligned interactions matrices by replicate are highly correlated (Spearman's $\rho \geq 0.69$)  % FIXME p-value!
for both cell lines, and significantly different from random (Appendix III).  The

% TODO Do the python statistics here to validate that the data is good.  This includes:
% 1) Spearman's between H1 and IMR90
% 2) Spearman's between H1 replicates
% 3) Compute a "random" Hi-C map and take correlation between these
% 4) Show that normalization does not significantly decrease the correlation
% 5) Include p-values for everything I state


\subsection*{DNA Methylation Analysis}

% TODO

\subsection*{Gene Expression and Pathway Analysis}

Affymetrix gene expression microarray data for IMR90 and H1 hESCs were obtained from GEO (GSE2672\cite{kim2005}
and GSE54186\cite{kim2014}, respectively).  Probe annotations were downloaded from the Affymetrix site for
the Human Genome U133v2 array, and remapped to genes from UCSC\@.  Noticing that expression profiles for replicates
were similar, the gene expression levels for each cell line were calculated by averaging across replicates, and
taking the log2 transform of the averaged signal, yielding signal for $23,100$ genes.

% TODO Kolmogorov-Smirnov test w/ normal distribution

To understand expression changes during differentiation from stem cell to lung cell, I subtracted the lung expression
signal from the stem cell line.  Taking the log2 transform yields the fold change in gene expression level per gene.
Two groups of genes were selected for further analysis: the 100 most upregulated genes and the 100 most downregulated
genes.  I was curious to find out whether classes of genes were changing together.  Using the ConsensusPathDB
tool\cite{kamburov2012}, I performed over-representation analysis for pathways and gene ontologies.

\subsection*{Histone Binding}

% TODO

\subsection*{Chromatin Accessibility Assay}

% TODO

Additionally, analyzed tracks for DNA methylation at base pair
resolution were obtained from GEO\cite{lister2009}.  The liftOver tool from
the University of California Santa Cruise was used to update the methylation
data from human genome build hg18 to hg19 for meaningful comparison\cite{hinrichs2006}.
For detailed data collection notes, see Appendix I.

The detailed description of the treatment protocol used in creating the Hi-C
library can be found in the original paper\cite{ren2013}.  Briefly, the IMR90
cells were cultured in growth media.

The cell lines were prepared with no treatment according to the protocol
described by Dekker and colleagues\cite{dekker2013}.  In brief, the cells
are washed with a formaldehyde fixation solution to cross-links DNA and
protein molecules near each other in genomic space.  The cells are
lysed and the nuclear contents are treated with a non-specific restriction
enzyme, \textit{HindIII}.  After restriction, the remaining nuclear material
is washed so that the remaining contents are short DNA sequences or cross
-linked DNA-protein complexes.  The DNA reads are labeled with a biotin
marker and ligated at low concentrations that favor within ligation between
reads bound to the same complex.  The reads are captured and sequenced,
building what is known as a Hi-C library.

%% Regulatory Elements from Ensembl

%% Translocations from COSMIC

%% Genes from Biomart
%% Hg19 from UCSC


\section*{Mapping and Alignment}

A Hi-C library is formed largely of \textit{chimeric} DNA sequences,
sequences composed of material from more than one source.  The ligation process
creates novel DNA sequences with a portion of sequence from one region of
the genome and the remaining portion from a different genomic region.  In
order to discover to which two regions each probe corresponds, the probes
must be realigned to the human genome.  Previous methods (\textit{citation needed})
have truncated reads to a fixed cutoff size before realignment with a short
read alignment algorithm.  In a recent paper, Imakaev and colleagues
describe an algorithm to repeated align reads with different truncation sizes
and analyzing reads with the maximum alignment score\cite{imakaev2012}.  We followed
this procedure to realign the IMR90 Hi-C libraries for six replicates to
the human genome build 19 (CRCh38/hg19), with an average of XX million reads
mapped per replicate.

\section*{Normalization of Contact Maps}

Understanding and accounting the biases inherent in a data set is vital to
producing accurate, biologically relevant analysis.  To this end, several methods
have been proposed identify biases and normalize contact maps between replicates
and experiments\cite{yaffe2011}\cite{hu2012}\cite{yang2014}.  The earliest methods on low
resolution samples did not perform any normalization at all\cite{aiden2009};
however, as higher resolution data sets have come available, better
normalization methods are being explored to identify reproducible,
fine-grained chromatin structures.

Normalizing data from Hi-C experiments is quite challenging.  As previously
discussed, the Hi-C data sets are heterogeneous in nature, incorporating
interactions from cells in every position of cell cycle.  As evidence of phenomena,
the interaction matrices produced by the Hi-C experiments have nonzero contact
probabilities between nearly every genomic coordinate\cite{dekker2013}.  This
finding is further reinforced by the observation that single cell Hi-C analysis
shows high variability in chromatin architecture\cite{nagano2013}.  Despite this
variability, existing normalization techniques for Hi-C data sets typically
assume the majority of the cells in the sampled ensemble are in a static
chromatin conformation reflective of \textit{in vivo} cell populations.

One approach to normalization described by Yaffe and colleagues\cite{yaffe2011}
attempts to describe the total interaction frequency between loci as the product
of a true interaction frequency and a set of experimental biases.  These biases
are present in all chromosomal capture techniques.  They include biases due to
PCR in efficiency, read depth per region, sequencing bias, chromatin compaction,
and random background collisions during ligation\cite{benner2014}\cite{dekker2006}.

\section*{Gene Expression Profiling}

We obtained total gene expression data for IMR90 and \gls{hESC} lines from \gls{GEO}.  The Affymetrix probe annotations were
downloaded from the Affymetrix website and reads were aligned to UCSC genes.  Probes without a gene match were discarded.  Raw
expression values were converted to log fold change values for each replicate.  The replicates were highly correlated ($\mu = 2$)
and expression histograms revealed high similarity in total gene expression between replicates.


The set of genes expressed in a cell, collectively known as the \gls{transcriptome}, provides insight into chromatin accessibility
and cell state.  A gene expressed in $n$ conditions (or $n$ samples) forms vector in $n-$dimensional space.  Similarly, the expression
profile the entire transcriptome for a single sample is a vector in $m-$dimensional space, where $m$ is the number of genes assessed.
We therefore denote an experiment interrogating $n$ genes across $m$ samples as an $m \times n$ matrix $S_{m,n}$.
\[
  S_{m,n} = \left[
    \begin{array}{cccc}
      s_{1,1} & s_{1,2} & \cdots & s_{1,n} \\
      s_{2,1} & s_{2,2} & \cdots & s_{2,n} \\
      \vdots & \vdots & \ddots  & \vdots \\
      s_{m,1} & s_{m,2} & \cdots & s_{m,n}
    \end{array}
  \right]
\]

and a single gene's expression profile is
\[ G_{i} = \begin{pmatrix} g_{i1} & g_{i2} & g_{i3} & \cdots & g_{in} & \end{pmatrix} \]

We wish to compare differences in gene expression profiles for particular, or all, genes between samples.  This amounts to comparing
\textit{distances} between gene expression vectors.  There are several methods for doing so.

\subsection*{Euclidean Distance}

The simplest genomic distance measure is \textbf{Euclidean} distance.  Euclidean distance is a particular \gls{norm}, known as
the 2-norm.  Geometrically, the norm is a measure of `size' or `proximity' in $n-$dimensional space.  The 2-norm for a vector $x$ is
defined as
\[
  \|x\|_{2} = {(|x_1|^2 + |x_2|^2 + \cdots + |x_n|^2)}^{\frac{1}{2}}
\]

The distance between two vectors in a vector space can now be found using the triangle inequality on the 2-norm.  As a norm, the Euclidean
norm has some desirable properties, notably, non-negativity, positivity, and a definition of the triangle inequality
$\|x + y\| = \|x\| + \|y\|$~\cite{horn2013}.  Directly from the triangle inequality, we define the Euclidean distance in $n-$dimensions to
be
\[
  D_{Euc}(x,y) = \|x\| - \|y\| = {(\sum_{i = 1}^{n}{(a_i - b_i)}^2)}^{\frac{1}{2}}
\]


