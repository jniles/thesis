\chapter{Methods}

The methods used in our analyses depend on publicly available open-source packages and custom scripts developed in house.
These scripts are permissively licensed under the GNU public license (version 3) and
\href{https://github.com/New-College-of-Florida/Jonathan-Niles-Thesis}{available online}.  Hi-C analysis was conducted using
the excellent hiclib from the lab of Leonid Mirny \citep{imakaev2012}.  Visualizations were created in the R programming
language \citep{r2009} and the matplotlib python package \citep{hunter2007}.  Genome distance calculations leveraged the
bedtools suite \citep{quinlan2010} and pybedtools package \citep{dale2011}.

\section*{Data Collection and Processing}

IMR90 lung fibroblast and H1 \glsentryfull{hESC} cell lines were examined due to the availability of data and body of preexisting
literature.  Hi-C data sets for both cell lines were obtained from \glsentryfull{GEO} \citep{edgar2002}, accession number
\href{http://www.ncbi.nlm.nih.gov/geo/query/acc.cgi?acc=GSE43070}{GSE43070}.  Two and six technical replicates were examined for
\gls{hESC} and IMR90 cell lines respectively.  Chimeric HI-C reads were aligned to the human genome build 19 using the BowTie2
alignment software\footnote{\url{http://bowtie-bio.sourceforge.net/bowtie2/index.shtml}} \citep{langmead2012}. Reads containing
segments mapping to unique portions of the genome were stored on disk in the high performance
HDF5\footnote{\url{http://www.hdfgroup.org/HDF5/}} format using the \href{https://bitbucket.org/mirnylab/hiclib}{hiclib} python
library \citep{imakaev2012}.  Low-scoring or single sided reads were discarded from future analysis.  Interaction maps were
stored in 2Mb, 1Mb, 200kb, 40kb, 20kb, and 10kb bin sizes.

\section*{Iterative Correction and Eigenvector Decomposition}

Exported maps were corrected using the `correctIC' function from hiclib.  The heatmaps were iteratively corrected as previously
described \citep{imakaev2012}, at an average of 10 iterations per map.  Pearson correlations were calculated between each sample
for both replicates.  The maps were decomposed into eigenvector components by \gls{PCA} using the `doPCA' functions in hiclib.
\gls{PCA} was performed on interaction maps of 1Mb and 200kb bin sizes.

\section*{Gene Expression and Pathway Analysis}

Gene expression data sets were obtained from \gls{GEO} (\href{http://www.ncbi.nlm.nih.gov/geo/query/acc.cgi?acc=GSE2672}{GSE2672},
\href{http://www.ncbi.nlm.nih.gov/geo/query/acc.cgi?acc=GSE54186}{GSE54186}) for each cell line \citep{kim2005} \citep{kim2014}.  Probe
annotations for the Affymetrix GeneChip Human Genome U133 Plus 2.0 microarray and gene annotations for hg19 were obtained from
Biomart \citep{kasprzyk2011}.  Upregulated genes were determined by taking a $\log_2$ ratio of the mean signal intensity between
replicates

\begin{align}
  \delta_i = \log_2{\frac{\bar{E_i^I}}{\bar{E_i^H}}}
\end{align}

where $\delta_i$ is the expression change for a gene $i$, $\bar{E_i^I}$ and $\bar{E_i^H}$ are the mean expression values across
IMR90 and hESC replicates, respectively.  Pathway analysis was performed using the online utilities provided by
\href{http://consensuspathdb.org/}{ConsensusPathDB} \citep{kamburov2012}.

\section*{Directionality Index and Domain Discovery}

The directionality index was calculated as previously described by Dixon and colleagues \citep{dixon2012}.  We used the highest
resolution (10kb) map generated by \gls{ICE} to compute the directionality index for 100kb, 200kb, 400kb, 800kb, 1Mb windows
genome wide.  We reasoned that domains exist between indices where a sign change occurs of the directionality index.  We located
regions of the directionality index with the 10\% largest peaks, and defined a topological domain to be the entire positive (negative)
region beneath (above) the index between sign changes.   `True' domains were determined as conserved domains between replicates.  The
The domains were exported to \.bed format for analysis using the pybedtools package \citep{quinlan2010,dale2011}.

\section*{Comparison to Cancerous Lesions}

Lung cancer data sets were obtained from the \gls{TCGA} for
\href{https://tcga-data.nci.nih.gov/tcga/tcgaCancerDetails.jsp?diseaseType=LUAD&diseaseName=Lung\%20adenocarcinoma}{514 cases}
of lung adenocarcinoma \citep{cerami2012,gao2013}.  The number of intersections between lesions and conserved domains were computed.

